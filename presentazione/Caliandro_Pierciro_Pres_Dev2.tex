\documentclass[10pt]{beamer}
\usepackage{hyperref}
\usepackage{xcolor}
\usepackage[export]{adjustbox} %used to align the images
\usepackage{tabularx}
\usepackage{lmodern,textcomp}
\usepackage[font=tiny]{caption}

\usetheme{Berlin}
\title{Deliverable 2: Applicazione di tecniche di ML su progetti open source}
%\subtitle{}
\author{Pierciro Caliandro}
\institute{Università degli studi di Roma Tor Vergata}

\definecolor{blue}{rgb}{0.0, 0.0, 1.0}
\definecolor{calpolypomonagreen}{rgb}{0.12, 0.3, 0.17}

\setbeamercolor{palette primary}{bg=calpolypomonagreen, fg=white}
%\setbeamercolor{palette secondary}{bg=calpolypomonagreen, fg=white}

\usecolortheme[named=blue]{structure}


\begin{document}
\footnotesize
\begin{frame}
\titlepage
\end{frame}

\begin{frame}
\frametitle{Introduzione}
Lo scopo della presentazione è quello di mostrare i risultati a seguito dell'applicazione di tecniche di sampling, classificazioni
sensibili al costo, e feature selection su modelli di ML. In particolare, ci si concentra su come tali tecniche impattano sulle metriche di accuratezza per i seguenti classificatori:
\begin{itemize}
\item 
\end{itemize}
\end{frame}
\end{document}